
\chapter{Objetivos}
\label{chap:objs}

\section{Objetivo General}
\label{sec:objGen}
Proponer una metodolog�a para analizar y organizar la informaci�n procedente de la comunicaci�n entre el entorno y el cerebro a trav�s del lazo sensoriomotor, que permita extraer caracter�sticas naturales de locomoci�n de comportamientos ya aprendidos, con el fin de formar modelos internos que permitan estructurar controladores y estimadores del comportamiento de su morfolog�a e interacci�n con el entorno de forma predictiva.

\section{Objetivos Espec�ficos}
\label{sec:objEsp}

\begin{itemize}
\item Construir un modelo computacional de un humanoide con un sistema NMS, sobre el cual se pueda aprender comportamientos por auto-organizaci�n y demostraci�n, con la finalidad de formar un generador de muestras estadisticas, de comportamientos humanoides para analizar.
\item Construir un controlador jer�rquico, que aprenda modelos internos del lazo sensoriomotor y que mejore su desemple�o a medida que interact�e con su entorno.
\item Proponer un metodo de transferencia del controlador NMS a una arquitectura humanoide
%\item Utilizar un modelo del sistema NMS para representar el lazo sensoriomotor de un humanoide, sobre el cual se implemente estrategias de control de comportamientos coordinados auto-organizado o aprendidos por demostraci�n que cumplan el objetivo de un comportamiento.
%\item Extraer la funci�n de costo mediante IRL de comportamientos aprendidos y buscar una estrategia �ptima local del comportamiento aprendido mediante RL.
%\item Proponer modelos neuronales sobre el NMS que permitan representar modelos probabil�sticos del comportamiento del humanoide, que funciones como predictores de las acciones y estimadores del entorno.
\end{itemize}

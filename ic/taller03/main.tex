\documentclass[10pt,onecolumn,twoside,letterpaper]{article}
\usepackage[text={7.7in,9.5in},centering]{geometry}
\usepackage[spanish,es-nodecimaldot]{babel}

\usepackage{hyperref}

\usepackage{multicol}
\usepackage{wrapfig}

\usepackage{amsmath,amssymb,amsthm}

\usepackage{harvard}% bibliographystyle: apsr, agsm, dcu, kluwer, nederlands
\newcommand{\myreferences}{../../bib/library}
\usepackage{graphicx}
\graphicspath{{../../images/}}
\usepackage{amssymb}
\usepackage{fancyhdr}
\usepackage{color}
\usepackage{colortbl}
\definecolor{gray}{cmyk}{0.0,0.0,0.0,0.60}

\usepackage{float}

%\usepackage{auto-pst-pdf}
%\usepackage{pst-all}

\usepackage{mcode}
%\usepackage[numbered]{mcode}
%\usepackage{lipsum}

\pagestyle{fancy}
\fancyhf{}
\fancyhead[RO]{\small{\textcolor{gray}{\textsc{Hacia un framework de locomoci\'on b\'ipeda, evolutiva y flexible}}}}
\fancyhead[LO]{\includegraphics[scale=0.05]{unlogo.png}}
\fancyhead[LE]{\includegraphics[scale=0.05]{unlogo.png}\quad\small{\textcolor{gray}{\textsc{Control Inteligente 2015-01}}}}
\fancyhead[RE]{\small{\textcolor{gray}{\textsc{TALLER 03}}}}
\fancyfoot[CO,CE]{\thepage}
\fancyfoot[LO,RE]{\scriptsize{\textcolor{gray}{\emph{Version 0.1}}}}

\title{\vspace{-0.8cm}\includegraphics[scale=0.12]{unescudobn.png}\\\vspace{-0.0cm}
  \LARGE \textbf{Taller 3 - Sistemas, Modelos Difusos y T\'ecnicas de Construcci\'on}}
\author{J.A. Castillo-Le\'on\thanks{jacastillol@unal.edu.co} \and Ronny Gelleschus\thanks{rgelleschus@unal.edu.co}}
\date{}

\begin{document}
\maketitle
\begin{enumerate}
\item Aquí utilizaría algo como en la presentación del profe (capítulo 3), página 14 en el documento que nos envió. Sin embargo para mí eso no es un diagrama de bloques, se tiene que cambiar los señales (y evitar el uso de un "BUS" de señales) y los símbolos de los elementos del sistema a bloques simples. No sé con cuál programa podemos hacerlo. Tú tienes una idea?

\item Algorithmo Mamdani (fuente: \textit{FNControl 2009.pdf}, página 37)\\
  Primero se tiene que calcular el grado de cumplimiento del conjunto en la entrada para cada regla. La fórmula general es $\beta_i = \max(\min(\mu_{A'}(x),\mu_{A_i}(x)))$. En el caso de una entrada concreta $x_0$ eso se simplifica a $\beta_i=\mu_{A_i}(x_0)$.\\
  En el segundo paso se calcula el conjunto que resulta para cada regla: $\mu_{B'_i}(y)=\beta_i\cdot \mu_{B_i}(y)$.\\
  Al final se agrega el resultado calculándo el máximo de todos los $\mu_{B'_i}$ en cada punto $y\in Y$: $\mu_{B'}(y)=\max(\mu_{B'_i}(y))$.\\
  Para una entrada $x_0=2$ y con las reglas establecidas en la tarea eso significa:
  \begin{enumerate}
  \item $\beta_1=0,6$ y $\beta_2=0,4$.
  \item $B'_1=[ 0,6/4; 0,6/5; 0,18/6]$ y $B'_2=[0,04/4; 0,36/5; 0,4/6]$.
  \item $B'=[0,6/4; 0,6/5; 0,4/6]$.
  \end{enumerate}

\item $z = \frac{\sum_{i=1}^{K}\beta_i\cdot z_i(x,y)}{\sum_{i=1}^{K} \beta_i}$, donde $\beta_i$ es el grado de cumplimiento de una regla y $K$ el número de reglas. Para calcular el grado de cumplimiento de cada regla $i$ se utilizó $\beta_i=\min(\mu_{A_i}(x),\mu_{B_i}(y))$. Entonces $z= \frac{0,1\cdot 6+0,9\cdot 7+0,1\cdot 14+0,1\cdot 7}{0,1+0,9+0,1+0,1}=7,5$.

\item Construcción de un modelo difuso basado en conocimiento (fuente: \textit{FNControl 2009.pdf}, página 79)\\
  \begin{enumerate}
  \item Selección de los variables de entrada y salida, la estructura de las reglas y los métodos de inferencia y defusificación
  \item Decisión cuántos términos linguísticos se tendrá en el sistema y definición de las funciones de membrencia correspondientes
  \item Formulación del conocimiento en forma de reglas If-Then
  \item Validación del modelo (normalmente usando valores reales disponibles). Si el modelo no cumple con las expectaciones iteración de los pasos anteriores.
  \end{enumerate}

\item Sistema dinámico singleton
  \begin{enumerate}
  \item Este sistema necesita un filtro dinámico en la entrada (?), una difusificación de los valores en la entrada y el sistema de inferencia con la base de reglas. Las reglas tendrán la forma\\
    $R^{i}$: \textbf{If} $y(k)$ is $A_{ii}$ \textbf{and} $u(k)$ is $B_{jj}$ \textbf{then} $y(k+1)=c_i$.\\Los índices $ii$ y $jj$ ambos son en el rango $[1;3]$ y varían independientemente. Se necesita al máximo 9 reglas.

  \item Los parámetros libres de este modelo son todos los parámetros de las funciones de pertenencia de $A_{ii}$ y $B_{jj}$ y los valores de los $c_i$.
  \end{enumerate}

\item No estoy seguro si también se tiene que incluir el $n_d$ en los términos de $y$, pero me imagino que sí porque se los usa para la parte autoregresiva (depende de si el elemento con el tiempo muerto está en el sistema o sólo en la medición del set point que tal vez viene de otro sistema?):\\ $y(k+1)=f_{NL}(y(k-n_d),y(k-n_d-1),\dots,y(k-n_d-n_y+1),u(k-n_d),u(k-n_d-1),\dots,u(k-n_d-n_u+1))$\\
  $y$ \dots valores de la salida (anteriores y futuro)\\
  $u$ \dots valores de la entrada anteriores\\
  $k$ \dots instante de tiempo actual\\
  $n_y$, $n_u$ \dots número de instantes anteriores que se tiene en cuenta para calcular $y(k+1)$\\
  $n_d$ número de instantes anteriores que no se puede utilizar por el tiempo muerto\\
  $f_{NL}$ \dots función (en general no lineal) que relaciona $y(k+1)$ con los valores anteriores de $y$ y $u$

\item (fuente: \textit{FNControl 2009.pdf}, página 89) Un modelo semi-mecanístico es un modelo que combina partes "caja blanca" y "caja negra". Así se puede evitar que se utiliza métodos de aproximación y identificación en relaciones físicos que se conoce exactamente.
  \begin{enumerate}
  \item ???
  \end{enumerate}
\end{enumerate}

\end{document}
